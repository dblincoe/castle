\documentclass[12pt]{article}
\usepackage{fullpage}
\usepackage[utf8]{inputenc}
\usepackage[T1]{fontenc}
\usepackage{microtype}
\usepackage{hyperref}
\usepackage{amssymb}
\usepackage{amsfonts}
\usepackage{float}
\usepackage{amsmath}
\usepackage{amsthm}
\usepackage{graphicx}
% \usepackage[backend=bibtex,style=ieee]{biblatex}
% \bibliography{citations}
\usepackage{wrapfig}
\usepackage{listings}
\usepackage{color}
\definecolor{dkgreen}{rgb}{0,0.6,0}
\definecolor{gray}{rgb}{0.5,0.5,0.5}
\definecolor{mauve}{rgb}{0.58,0,0.82}
\lstset{frame=tb,
  language=python,
  aboveskip=3mm,
  belowskip=3mm,
  showstringspaces=false,
  columns=flexible,
  basicstyle={\small\ttfamily},
  numbers=none,
  numberstyle=\tiny\color{gray},
  keywordstyle=\color{blue},
  commentstyle=\color{dkgreen},
  stringstyle=\color{mauve},
  breaklines=true,
  breakatwhitespace=true,
  tabsize=4
}
\graphicspath{{images/}}
\setlength{\parindent}{2em}
\setlength{\parskip}{0.25em}
\title{EECS 395 Project Proposal: Computer Algebra System}
\author{David Blincoe \and
Imran Hossain \and
Sam Jenkins \and
Kennan LeJeune \and
Chris Toomey \and Ben Young}
\date{\today}

\begin{document}
    \maketitle
    \section{Background and Project Goals}
      A computer algebra system (CAS) is a program that symbolically manipulates potentially complex mathematical formulae and outputs a solution. This helps automate equations that may be too difficult for humans to solve in a reasonable amount of time. This differs from a typical calculator in two main ways:
      \begin{enumerate}
        \item It has the ability to solve equations involving symbols in addition to numerics
        \item The user can store self-created equations and functions.
      \end{enumerate}
      Our goal for this project is to develop a computer algebra system and an interface for interacting with it. We hope to implement as many of the following features as possible from scratch, without using external math libraries.
    \section{Features}
      \begin{enumerate}
        \item Intuitive user interface for input of equations and output of solutions
        \item Simplify expressions
        \item Factor polynomials
        \item Factor integers into primes
        \item Symbolic manipulation of input equations

      \end{enumerate}

\end{document}